\input mftmac
\def\9{$\futurelet\next\doit}
\let\[=\relax
\def\doit{\ifx\next\[\def\next{\hfill{\rm\%}}%
     \else\ifx\next\ \def\next{\qquad{\rm\%}}%
     \else\def\next{{\rm\%}}\fi\fi\next}
\input epsf
\def\newpage{\vfill\eject}
\def\vc#1{$\vcenter{#1}$}
\def\title#1{\hrule\vskip1mm#1\par\vskip1mm\hrule\vskip5mm}
\def\figure#1{\par\centerline{\epsfbox{#1}}}
\title{{\bf POLY.MP: DEFINITION OF REGULAR CONVEX POLYHEDRA IN METAPOST}}
\BL
 version 1.0, 8 April 1998
 {\bf Denis Roegel} ({\tt roegel@loria.fr})

This package defines the five regular convex polyhedra classes and
must be used with the {\bf 3d} package.
\BL
Each object definition is parameterized by
an instance identification.
\BL
All polyhedra are normalized (centered on the origin and inscriptible
in a sphere of radius 1).
\BL
The polyhedra are defined with a certain number of points and faces.
The points comprise both the vertices and the center of the object.
\BL
It might be useful to remind the Euler formula linking
the number of faces $f$,
vertices $v$ and edges $e$ in polyhedra with no hole in dimension 3:
$f+v-e=2$. This can be verified in the following table:
\BL
\centerline{\vbox{\halign{\quad#\hfil\quad&&\quad\hfil#\quad\cr
 \omit \hfil type\hfil&\omit\hfil faces\hfil
                      &\omit\hfil vertices\hfil&\omit\hfil edges\hfil\cr
 tetrahedron&4&4&6\cr
 cube&6&8&12\cr
 octahedron&8&6&12\cr
 dodecahedron&12&20&30\cr
 icosahedron&20&12&30\cr}}}
\BL
For more details, see the documentation enclosed in the distribution.
\BL
$\2{message}\7"*** poly, version 1.0 (c) D. Roegel, 8 April 1998 ***";$\par
\BL
\newpage\title{TETRAHEDRON (4 faces)\kern1cm\epsfbox{tetra.ps}}
\BL
 Construction of a tetrahedron:
\BL
 One face is \vc{\epsfbox{vect-fig.1}} with $h=c\sqrt{3}/2$
\BL
 The angle $\alpha$ between two faces is given by
 $\sin(\alpha/2)={\displaystyle c/2\over \displaystyle c\sqrt{3}/2}=1/\sqrt3$
\BL
 \figure{vect-fig.2}
\BL
$\2{def}\\{set\_tetrahedron\_points}(\2{expr}\\{inst})=$\par
\quad\quad$\\{set\_obj\_point}(1,0,0,0);\ \\{set\_obj\_point}(2,1,0,0);$\par
\quad\quad$\\{set\_obj\_point}(3,\1{cosd}(60),\1{sind}(60),0);\ \9 $60=360/6$%
\par
\quad\quad$\\{sinan}=1/\1{sqrt}(3);\ \\{cosan}=\1{sqrt}(1-\\{sinan}\6{\ast \ast
}2);$\par
\quad\quad$\\{an}=180-2\ast \1{angle}((\\{cosan},\\{sinan}));$\par
\quad\quad$\\{new\_face\_point}(4,1,2,3,\\{an});$\par
\quad\quad$\\{normalize\_obj}(\\{inst})(1,2,3,4);$\par
\quad\quad$\\{set\_obj\_point}(5,0,0,0);\ \9 center of tetrahedron\par
$\!\3{enddef};$\par
\BL
$\2{vardef}\\{def\_tetrahedron}(\2{expr}\\{inst})=$\par
\quad\quad$\\{new\_obj\_points}(\\{inst},5);\ \9 4 vertices and the center\par
\quad\quad$\\{new\_obj\_faces}(\\{inst},4);\ \9 4 faces in the definition\par
\quad\quad$\\{set\_tetrahedron\_points}(\\{inst});$\par
\quad\quad$\\{set\_obj\_face}(1,\7"1,2,4",\7"b4fefe");$\par
\quad\quad$\\{set\_obj\_face}(2,\7"2,3,4",\7"b49bc0");$\par
\quad\quad$\\{set\_obj\_face}(3,\7"1,4,3",\7"b4c8fe");$\par
\quad\quad$\\{set\_obj\_face}(4,\7"1,3,2",\7"b4fe40");$\par
$\!\3{enddef};$\par
\BL
\newpage\title{CUBE (HEXAHEDRON) (6 faces)\kern1cm\epsfbox{cube.ps}}
\BL
The definition of a cube is straightforward.
\BL
$\2{def}\\{set\_cube\_points}(\2{expr}\\{inst})=$\par
\quad\quad$\\{set\_obj\_point}(1,0,0,0);$\par
\quad\quad$\\{set\_obj\_point}(2,1,0,0);$\par
\quad\quad$\\{set\_obj\_point}(3,1,1,0);$\par
\quad\quad$\\{set\_obj\_point}(4,0,1,0);$\par
\quad\quad$\\{set\_obj\_point}(5,0,0,1);$\par
\quad\quad$\\{set\_obj\_point}(6,1,0,1);$\par
\quad\quad$\\{set\_obj\_point}(7,1,1,1);$\par
\quad\quad$\\{set\_obj\_point}(8,0,1,1);$\par
\quad\quad$\\{normalize\_obj}(\\{inst})(1,7);\ \9 1 and 7 are opposite vertices%
\par
\quad\quad$\\{set\_obj\_point}(9,0,0,0);\ \9 center of cube\par
$\!\3{enddef};$\par
\BL
$\2{vardef}\\{def\_cube}(\2{expr}\\{inst})=$\par
\quad\quad$\\{new\_obj\_points}(\\{inst},9);\ \9 8 vertices and the center\par
\quad\quad$\\{new\_obj\_faces}(\\{inst},6);\ \9 6 faces in the definition\par
\quad\quad$\\{set\_cube\_points}(\\{inst});$\par
\quad\quad$\\{set\_obj\_face}(1,\7"1,2,6,5",\7"b4fe40");$\par
\quad\quad$\\{set\_obj\_face}(2,\7"2,3,7,6",\7"45d040");$\par
\quad\quad$\\{set\_obj\_face}(3,\7"4,8,7,3",\7"45a114");$\par
\quad\quad$\\{set\_obj\_face}(4,\7"1,5,8,4",\7"45a1d4");$\par
\quad\quad$\\{set\_obj\_face}(5,\7"5,6,7,8",\7"4569d4");$\par
\quad\quad$\\{set\_obj\_face}(6,\7"4,3,2,1",\7"112da1");$\par
$\!\3{enddef};$\par
\BL
\BL
\newpage\title{OCTAHEDRON (8 faces)\kern1cm\epsfbox{octa.ps}}
\BL
 A section of a pyramid is: \vc{\epsfbox{vect-fig.3}}
\BL
 The height of this pyramid is given
 by $H^2={(c\sqrt3/2)}^2-{(c/2)}^2=c^2/2$, hence $H=c/\sqrt2$
\BL
$\2{def}\\{set\_octahedron\_points}(\2{expr}\\{inst})=$\par
\quad\quad$\\{set\_obj\_point}(1,0,0,0);$\par
\quad\quad$\\{set\_obj\_point}(2,1,0,0);$\par
\quad\quad$\\{set\_obj\_point}(3,1,1,0);$\par
\quad\quad$\\{set\_obj\_point}(4,0,1,0);$\par
\quad\quad$\\{set\_obj\_point}(5,.5,.5,1/\1{sqrt}(2));$\par
\quad\quad$\\{set\_obj\_point}(6,.5,.5,-1/\1{sqrt}(2));$\par
\quad\quad$\\{normalize\_obj}(\\{inst})(5,6);\ \9 5 and 6 are opposite vertices%
\par
\quad\quad$\\{set\_obj\_point}(7,0,0,0);\ \9 center of octahedron\par
$\!\3{enddef};$\par
\BL
$\2{vardef}\\{def\_octahedron}(\2{expr}\\{inst})=$\par
\quad\quad$\\{new\_obj\_points}(\\{inst},7);\ \9 6 vertices and the center\par
\quad\quad$\\{new\_obj\_faces}(\\{inst},8);\ \9 8 faces in the definition\par
\quad\quad$\\{set\_octahedron\_points}(\\{inst});$\par
\quad\quad$\\{set\_obj\_face}(1,\7"1,2,5",\7"b4fefe");$\par
\quad\quad$\\{set\_obj\_face}(2,\7"2,3,5",\7"45d040");$\par
\quad\quad$\\{set\_obj\_face}(3,\7"3,4,5",\7"4569d4");$\par
\quad\quad$\\{set\_obj\_face}(4,\7"4,1,5",\7"b49bc0");$\par
\quad\quad$\\{set\_obj\_face}(5,\7"6,1,4",\7"45a1d4");$\par
\quad\quad$\\{set\_obj\_face}(6,\7"6,2,1",\7"b4c8fe");$\par
\quad\quad$\\{set\_obj\_face}(7,\7"6,3,2",\7"b49b49");$\par
\quad\quad$\\{set\_obj\_face}(8,\7"6,4,3",\7"112da1");$\par
$\!\3{enddef};$\par
\BL
\BL
\newpage\title{DODECAHEDRON (12 faces)\kern1cm\epsfbox{dodeca.ps}}
\BL
 Two adjacent faces (pentagons) of the dodecahedron are as follows:
\figure{vect-fig.13}
 The angle between the faces is the angle between $\overrightarrow{OA}$
 and $\overrightarrow{OB}$.
 $OA=OB=d_3$ as defined in \figure{vect-fig.14}
 $d_1=2r\sin(\pi/5)$, $d_2=2r\sin(2\pi/5)$
 and $d_3=d_1\sin(2\pi/5)=2r\sin(\pi/5)\sin(2\pi/5)$

 The angle $\alpha$ is defined by the following conditions:
 \figure{vect-fig.15}
 Thus $d_2^2=d_3^2+d_3^2-2d_3^2\cos\alpha$, which leads to
 $\cos\alpha=1-{\displaystyle d_2^2\over\displaystyle2d_3^2}
      =-{\displaystyle\cos(2\pi/5)\over\displaystyle2\sin^2(\pi/5)}$
\newpage
\BL
$\2{def}\\{set\_dodecahedron\_points}(\2{expr}\\{inst})=$\par
\quad\quad$\\{new\_points}(\\{fc})(10);\ \9 face centers\par
\quad\quad$\\{set\_point}(\\{fc}_{1},0,0,0);\ \\{set\_obj\_point}(1,1,0,0);$\par
\quad\quad$\\{set\_obj\_point}(2,\1{cosd}(72),\1{sind}(72),0);\ \9 72=360/5\par
\quad\quad$\\{rotate\_in\_plane}(3,\\{fc}_{1},1,2);$\par
\quad\quad$\\{rotate\_in\_plane}(4,\\{fc}_{1},2,3);$\par
\quad\quad$\\{rotate\_in\_plane}(5,\\{fc}_{1},3,4);$\par
\quad\quad$\\{cosan}=-\1{cosd}(72)/(2\ast \1{sind}(36)\ast \1{sind}(36));\ %
\\{sinan}=\1{sqrt}(1-\\{cosan}\6{\ast \ast }2);$\par
\quad\quad$\\{an}=180-\1{angle}((\\{cosan},\\{sinan}));$\par
\quad\quad$\\{new\_abs\_face\_point}(\\{fc}_{2},\\{fc}_{1},1,2,\\{an});$\par
\quad\quad$\\{new\_abs\_face\_point}(\\{fc}_{3},\\{fc}_{1},2,3,\\{an});$\par
\quad\quad$\\{new\_abs\_face\_point}(\\{fc}_{4},\\{fc}_{1},3,4,\\{an});$\par
\quad\quad$\\{new\_abs\_face\_point}(\\{fc}_{5},\\{fc}_{1},4,5,\\{an});$\par
\quad\quad$\\{new\_abs\_face\_point}(\\{fc}_{6},\\{fc}_{1},5,1,\\{an});$\par
\quad\quad$\\{rotate\_in\_plane}(6,\\{fc}_{2},2,1);$\par
\quad\quad$\\{rotate\_in\_plane}(7,\\{fc}_{2},1,6);$\par
\quad\quad$\\{rotate\_in\_plane}(8,\\{fc}_{2},6,7);$\par
\quad\quad$\\{rotate\_in\_plane}(9,\\{fc}_{3},2,8);$\par
\quad\quad$\\{rotate\_in\_plane}(10,\\{fc}_{3},8,9);$\par
\quad\quad$\\{rotate\_in\_plane}(11,\\{fc}_{4},3,10);$\par
\quad\quad$\\{rotate\_in\_plane}(12,\\{fc}_{4},10,11);$\par
\quad\quad$\\{rotate\_in\_plane}(13,\\{fc}_{5},4,12);$\par
\quad\quad$\\{rotate\_in\_plane}(14,\\{fc}_{5},12,13);$\par
\quad\quad$\\{rotate\_in\_plane}(15,\\{fc}_{6},5,14);$\par
\quad\quad$\\{new\_abs\_face\_point}(\\{fc}_{7},\\{fc}_{2},6,7,\\{an});$\par
\quad\quad$\\{new\_abs\_face\_point}(\\{fc}_{8},\\{fc}_{3},8,9,\\{an});$\par
\quad\quad$\\{new\_abs\_face\_point}(\\{fc}_{9},\\{fc}_{4},10,11,\\{an});$\par
\quad\quad$\\{new\_abs\_face\_point}(\\{fc}_{10},\\{fc}_{5},12,13,\\{an});$\par
\quad\quad$\\{rotate\_in\_plane}(16,\\{fc}_{7},6,15);$\par
\quad\quad$\\{rotate\_in\_plane}(17,\\{fc}_{7},15,16);$\par
\quad\quad$\\{rotate\_in\_plane}(18,\\{fc}_{8},7,17);$\par
\quad\quad$\\{rotate\_in\_plane}(19,\\{fc}_{9},9,18);$\par
\quad\quad$\\{rotate\_in\_plane}(20,\\{fc}_{10},11,19);$\par
\quad\quad$\\{normalize\_obj}(\\{inst})(2,20);\ \9 opposite vertices\par
\quad\quad$\\{set\_obj\_point}(21,0,0,0);\ \9 center of dodecahedron\par
\quad\quad$\\{free\_points}(\\{fc})(10);$\par
$\!\3{enddef};$\par
\newpage
$\2{vardef}\\{def\_dodecahedron}(\2{expr}\\{inst})=$\par
\quad\quad$\2{save}\\{cosan},\\{sinan},\\{an};$\par
\quad\quad$\\{new\_obj\_points}(\\{inst},21);\ \9 21 points in the definition%
\par
\quad\quad$\\{new\_obj\_faces}(\\{inst},12);\ \9 12 faces in the definition\par
\quad\quad$\\{set\_dodecahedron\_points}(\\{inst});$\par
\quad\quad$\\{set\_obj\_face}(1,\7"5,4,3,2,1",\7"ff0fa1");$\par
\quad\quad$\\{set\_obj\_face}(2,\7"8,7,6,1,2",\7"b40000");$\par
\quad\quad$\\{set\_obj\_face}(3,\7"10,9,8,2,3",\7"b49b49");$\par
\quad\quad$\\{set\_obj\_face}(4,\7"12,11,10,3,4",\7"b49bc0");$\par
\quad\quad$\\{set\_obj\_face}(5,\7"14,13,12,4,5",\7"b4c8fe");$\par
\quad\quad$\\{set\_obj\_face}(6,\7"6,15,14,5,1",\7"b4fefe");$\par
\quad\quad$\\{set\_obj\_face}(7,\7"6,7,17,16,15",\7"b4fe40");$\par
\quad\quad$\\{set\_obj\_face}(8,\7"8,9,18,17,7",\7"45d040");$\par
\quad\quad$\\{set\_obj\_face}(9,\7"10,11,19,18,9",\7"45a114");$\par
\quad\quad$\\{set\_obj\_face}(10,\7"11,12,13,20,19",\7"45a1d4");$\par
\quad\quad$\\{set\_obj\_face}(11,\7"14,15,16,20,13",\7"4569d4");$\par
\quad\quad$\\{set\_obj\_face}(12,\7"16,17,18,19,20",\7"112da1");$\par
$\!\3{enddef};$\par
\BL
\newpage\title{ICOSAHEDRON (20 faces)\kern1cm\epsfbox{icosa.ps}}
\BL
Two faces of an icosahedron are linked in the following way:
\BL
 \vc{\epsfbox{vect-fig.4}} where $h=c\sqrt3/2$
\BL
 $d$ is also a diagonal in a pentagon: \vc{\epsfbox{vect-fig.5}}
\BL
 $c=2r\sin(\pi/5)$,
 $d=2r\sin(2\pi/5)=2c\cos(\pi/5)$
\BL
 The angle between two faces is computed as follows:
 \vc{\epsfbox{vect-fig.6}}
\BL
 We have ${(2\cos(\pi/5))}^2=2{(\sqrt3/2)}^2-2{(\sqrt3/2)}^2\cos\alpha$
\BL
 Thus $cos\alpha=1-{8\over3}\cos^2(\pi/5)$
\newpage
\BL
$\2{def}\\{set\_icosahedron\_points}(\2{expr}\\{inst})=$\par
\quad\quad$\\{set\_obj\_point}(1,0,0,0);\ \\{set\_obj\_point}(2,1,0,0);$\par
\quad\quad$\\{set\_obj\_point}(3,\1{cosd}(60),\1{sind}(60),0);\ \9 60=360/6\par
\quad\quad$\\{cosan}=1-\frac8/{3}\ast \1{cosd}(36)\ast \1{cosd}(36);\ %
\\{sinan}=\1{sqrt}(1-\\{cosan}\6{\ast \ast }2);$\par
\quad\quad$\\{an}=180-\1{angle}((\\{cosan},\\{sinan}));$\par
\quad\quad$\\{new\_face\_point}(4,1,2,3,\\{an});$\par
\quad\quad$\\{new\_face\_point}(5,2,3,1,\\{an});$\par
\quad\quad$\\{new\_face\_point}(6,3,1,2,\\{an});$\par
\quad\quad$\\{new\_face\_point}(7,2,4,3,\\{an});$\par
\quad\quad$\\{new\_face\_point}(8,3,5,1,\\{an});$\par
\quad\quad$\\{new\_face\_point}(9,1,6,2,\\{an});$\par
\quad\quad$\\{new\_face\_point}(10,3,4,7,\\{an});$\par
\quad\quad$\\{new\_face\_point}(11,3,7,5,\\{an});$\par
\quad\quad$\\{new\_face\_point}(12,1,8,6,\\{an});$\par
\quad\quad$\\{normalize\_obj}(\\{inst})(1,10);\ \9 opposite vertices\par
\quad\quad$\\{set\_obj\_point}(13,0,0,0);\ \9 center of icosahedron\par
$\!\3{enddef};$\par
\BL
$\2{vardef}\\{def\_icosahedron}(\2{expr}\\{inst})=$\par
\quad\quad$\2{save}\\{cosan},\\{sinan},\\{an};$\par
\quad\quad$\\{new\_obj\_points}(\\{inst},13);\ \9 12 vertices and the center\par
\quad\quad$\\{new\_obj\_faces}(\\{inst},20);\ \9 20 faces in the definition\par
\quad\quad$\\{set\_icosahedron\_points}(\\{inst});$\par
\quad\quad$\\{set\_obj\_face}(1,\7"3,2,1",\7"b40000");$\par
\quad\quad$\\{set\_obj\_face}(2,\7"2,3,4",\7"ff0fa1");$\par
\quad\quad$\\{set\_obj\_face}(3,\7"3,7,4",\7"b49b49");$\par
\quad\quad$\\{set\_obj\_face}(4,\7"3,5,7",\7"b49bc0");$\par
\quad\quad$\\{set\_obj\_face}(5,\7"3,1,5",\7"b4c8fe");$\par
\quad\quad$\\{set\_obj\_face}(6,\7"1,8,5",\7"b4fefe");$\par
\quad\quad$\\{set\_obj\_face}(7,\7"1,6,8",\7"b4fe40");$\par
\quad\quad$\\{set\_obj\_face}(8,\7"1,2,6",\7"45d040");$\par
\quad\quad$\\{set\_obj\_face}(9,\7"2,9,6",\7"45a114");$\par
\quad\quad$\\{set\_obj\_face}(10,\7"2,4,9",\7"45a1d4");$\par
\quad\quad$\\{set\_obj\_face}(11,\7"9,4,10",\7"4569d4");$\par
\quad\quad$\\{set\_obj\_face}(12,\7"4,7,10",\7"112da1");$\par
\quad\quad$\\{set\_obj\_face}(13,\7"7,5,11",\7"b4fefe");$\par
\quad\quad$\\{set\_obj\_face}(14,\7"5,8,11",\7"b49bc0");$\par
\quad\quad$\\{set\_obj\_face}(15,\7"8,6,12",\7"45a114");$\par
\quad\quad$\\{set\_obj\_face}(16,\7"6,9,12",\7"b49b49");$\par
\quad\quad$\\{set\_obj\_face}(17,\7"8,12,11",\7"b40000");$\par
\quad\quad$\\{set\_obj\_face}(18,\7"7,11,10",\7"45a1d4");$\par
\quad\quad$\\{set\_obj\_face}(19,\7"12,10,11",\7"b4c8fe");$\par
\quad\quad$\\{set\_obj\_face}(20,\7"9,10,12",\7"ff0fa1");$\par
$\!\3{enddef};$\par
\BL
$\5{endinput}$\par
\BL
