\documentclass[10pt]{article}

\newcommand{\onearrow}[3]{\mbox{$#1 \stackrel{#2}{\longrightarrow} #3$}}
\newcommand{\calB}{\mbox{${\cal B}\ $}}
\newcommand{\calC}{\mbox{${\cal C}\ $}}
\newcommand{\calD}{\mbox{${\cal D}\ $}}
\newcommand{\calG}{\mbox{${\cal G}\ $}}
\newcommand{\calH}{\mbox{${\cal H}\ $}}
\newcommand{\calI}{\mbox{${\cal I}\ $}}
\newcommand{\calS}{\mbox{${\cal S}\ $}}
\newcommand{\functor}[3]{\mbox{${\cal #1} \stackrel{#2}{\rightarrow}{\cal #3}$}}
\newcommand{\mapto}[2]{\mbox{$#1 \mapsto #2$}}
\newcommand{\maptob}[3]{\mbox{$#1: #2 \mapsto #3$}}

\usepackage{amsthm}
\usepackage{amssymb}
\usepackage{amsmath,amscd}
\usepackage[all,cmtip]{xy}
\theoremstyle{remark}
\newtheorem{remark}{Remark}
\newtheorem{definition}{Definition}
\newtheorem{proposition}{Proposition}
\newtheorem{example}{Example}

\begin{document}

\section{Preliminaries}

\begin{definition}{$\mathbb{P}_{+} (S)$}
$$
\mathbb{P}_{+} (S) = \{ A \subseteq S \, | \, A \neq \emptyset \}
$$
\end{definition}

\begin{definition}[\textbf{Ordered power set} ($\mathbb{O}_+(S)$)]
$$
\mathbb{O}_+(S) = \coprod_{n \geq 1} S^{n}
$$
\end{definition}

\begin{definition}[\textbf{Hom Set Functor}]
Let \calC and \calD be categories. Let \onearrow{\calC}{F}{\calD} be a functor, and construct the functor  \onearrow{\mathbf{Sets}^{\calD}}{R_A}{\mathbf{Sets}^{\calC}} as follows:

\xymatrix{
                          & & & & G \circ A \ar@{(.>}[d] \\
\calC \ar[d]_A & C \ar[r]^{f} & C' & & \mathbf{Sets}^{\calC} \\
\calD                & A(C) \ar[r]^{A(f)} & A(C') & & \mathbf{Sets}^{\calD} \ar[u]_{R_A} \\
                         & & & & G \ar@{(.>}[u] \ar@{|.>}@/^2.5pc/[uuu]|{R_A(G)}
}

\xymatrix{
\onearrow{\calC}{F}{\mathbf{Sets}} \ar[d]_{\eta}^{\bullet} \\
\onearrow{\calD}{G}{\mathbf{Sets}} \\
}

\end{definition}

\begin{definition}[\textbf{Symmetric multi-relation}]
A \emph{Symmetric multi-relation} on a set $S$ is a subset $R \subseteq \mathbb{P}(S)$.
\end{definition}

\begin{definition}[\textbf{Closed under taking subsets}]
The symmetric multi-relation $R \subseteq \mathbb{P}(S)$ is \emph{closed under taking subsets} if
$(x \subseteq S \land x \in R) \implies x' \subseteq x \implies x' \in R$.
\end{definition}

\begin{definition}[\textbf{Non-Symmetric multi-relation}]
A \emph{Non-symmetric multi-relation} on a set $S$ is a subset $R \subseteq \mathbb{O}(S) = \coprod_{n \geq 1} S^{n}$.
\end{definition}

\begin{definition}[\textbf{tuple over} $S$]
Let $\bar{n}$ denote the set of natural numbers $\{0, \ldots, n\}$. Then an $n$-\emph{tuple} over $S$ is
a function \onearrow{\bar{n}}{f}{S}. We can loosen this definition by defining an $n$-tuple as a function from any ordered set of cardinality $n$ to $S$.
\end{definition}

\begin{remark}{Tuples subsets}
Given this definition of a \emph{tuple}, one can regard one tuple \onearrow{m}{f}{S} as a subset of another
tuple \onearrow{x}{g}{S} when $x$ is a subsequence of $y$.
\end{remark}

\begin{example}[Of a non-symmetric multi-relation]
Let $S=\{0,1,2,3\}$. Then $S^0 = \emptyset, S^1 = S = \{0,1,2,3\}, S^2 = S \times S$, and $S^3 = S \times S \times S$, and so on. Define the non-symmetric multi-relation $R$ on $S$ by
$$\xymatrix@R=1ex@C=1ex{
S^0            &  S^1 & S^2   & S^3 & S^4 & \ldots \\
\emptyset  & (1)    & (1,2) & (1,2,1) \\
                   & (2)    & (2,1) & (1,2,2) \\
                   & (3)    & (3,2) \\
                   &	    & (2,2)
}
$$
\end{example}

\begin{definition}[$\hom$-functor $\hom(-,b)$]
Let \calC and \calD be categories. Define the \emph{contravariant} $\hom$ \emph{functor} \onearrow{\calC}{\hom(-,b)}{\mathbf{Set}} as follows
$$
\xymatrix{
a \ar[d]_{g} & \hom(a,b) & f \circ g \ar@{(->}[l] \\
a'   		  & \hom(a',b) \ar[u]|{g^* = \hom(g,b)} & f \ar@{(->}[l] \ar[u]_{g^*}
}
$$
\begin{description}
\item [objects] \maptob{\hom(-,b)}{a}{\hom(a,b)}, i.e. the object $a$ is mapped to the \emph{set} of morphisms from $a$ to $b$.
\item [arrows] \maptob{\hom(-,b)}{\onearrow{a}{g}{a'}}{g^* = \hom(g,b)}, i.e. the function $g$ is mapped to the function $g^*$ defined by \maptob{g^*}{f}{f \circ g}.
\end{description}
\end{definition}

\section{Graphs, Hypergraphs and Simplicial Sets}

%%----------------------------------------------------------------------------------------------------------
\subsection{Graphs}

\begin{definition}[\textbf{Directed Graph}]
A \emph{directed graph} is a triple $X=(X_E, X_V, f)$ where \onearrow{X_E}{f}{X_V \times X_V}. The function $f$
is called the \emph{constituent function}.
\end{definition}

\begin{remark}[Directed Graph Representation]
A directed graph can be written
$\xymatrix{
     X_E \ar@<0.6ex>[r]^-{f_0} \ar@<-0.6ex>[r]_-{f_1} & X_V
}$
or equivalently as
$\xymatrix{
     X_E \ar[r]^-{f} & X_V \times X_V
}$
We will often drop the $X$ when it is understood from the context, as in
$\xymatrix{
     E \ar@<0.6ex>[r]^-{f_0} \ar@<-0.6ex>[r]_-{f_1} & V
}$
or equivalently as
$\xymatrix{
     E \ar[r]^-{f} & V \times V
}$
\end{remark}

\begin{definition}[Directed Graph Morphism]
Let $X=\onearrow{X_E}{f}{X_V \times X_V}$ and $Y=\onearrow{Y_E}{g}{\mathbb{O}_{+} (Y_V)}$ be directed graphs. A \emph{directed graph morphism} is a pair of functions \onearrow{X_E}{a}{Y_E} and \onearrow{X_V}{b}{Y_V} such that the following diagram commutes:
$$\xymatrix{
E \ar[d]^f & & E  \ar@<-0.5ex>[d]_{f_1} \ar@<0.5ex>[d]^{f_0} & & X_E  \ar[d]^f \ar[r]^-a  & Y_E  \ar[d]^g \\ 
V \times V & &V & &  X_V \times X_V   \ar[r]^-b       & Y_V \times Y_V
}$$
\end{definition}

\begin{definition}[\textbf{Index category for graphs} ($\Delta_G$)]
The index category for graphs consists of:
\begin{description}
\item [objects] $V$ and $E_2$
\item [arrows] \onearrow{V}{d_0}{E_2} and \onearrow{V}{d_1}{E_2}
\end{description}

This is diagramed as:
$\xymatrix{
     E_2 \ar@<1ex>[r]|-{d_0} \ar@<-1ex>[r]|-{d_1} & V
     }$
\end{definition}

\begin{remark}
$\Delta_G^{\text{op}}$ denotes the opposite category of $\Delta_G$.
The category $\Delta_G^{\text{op}}$ is diagramed as:
$\xymatrix{
     E_2 & V \ar@<1ex>[l]|-{d_0} \ar@<-1ex>[l]|-{d_1}
     }$
\end{remark}

\begin{proposition}[Categorical definition of graphs]
Let $\calG = \mathbf{Sets}^{\Delta_G^{\text{op}}}$ be the category of graphs.
A functor \onearrow{\Delta_G^{\text{op}}}{F}{\mathbf{Sets}} determines two sets $F(V)$ and $F(E_2)$, and functions
$\xymatrix{
     F(E_2) \ar@<1ex>[r]|-{d_0} \ar@<-1ex>[r]|-{d_1} & F(V)
     } $ which is the data that defines a graph. The opposite category determines the data is
$\xymatrix{
     F(E_2) & F(V) \ar@<1ex>[l]|-{d_0} \ar@<-1ex>[l]|-{d_1}
     }$ which is equivalent to
$\xymatrix{
     F(E_2) & F(V) \times F(V) \ar[l]|-{d}
     }$
so a natural transformation \onearrow{F}{\eta}{F'} determines functions \onearrow{F(V)}{\eta_V}{F'(V)} and \onearrow{F(E_2)}{\eta_{E_2}}{F(E_2)}.
$$
\xymatrix{ 
F(E_2) \ar@<1ex>[rr]|-{F(d_0)} \ar@<-1ex>[rr]|-{F(d_1)} \ar[d]_{\eta_{E_2}} & & F(V) \ar[d]^{\eta_V} & & F(E_2) \ar[r]^-{F(d)}  \ar[d]_{\eta_{E_2}} & F(V) \times F(V) \ar[d]^{\eta_V \times \eta_V} \\
F'(E_2) \ar@<1ex>[rr]|-{F'(d_0)} \ar@<-1ex>[rr]|-{F'(d_1)} & & F'(V) & & F'(E_2) \ar[r]^-{F'(d)} & F'(V) \times F'(V)
}
$$
\end{proposition}

%%----------------------------------------------------------------------------------------------------------
\subsection{Hypergraphs}

\begin{definition}[\textbf{Directed Hypergraph}]
A \emph{directed hypergraph} is a triple $X=(X_E, X_V, f)$ where \onearrow{X_E}{f}{\mathbb{O}_{+} (X_V)}.
Recall that $\mathbb{O}_+(X_V) = \coprod_{n \geq 1} X_V^{n}$. $f$ is called the \emph{constituent function} of the graph $X$.
\end{definition}

\begin{remark}
Equivalently, we can take $f$ to be a collection of functions $\{ \onearrow{E}{f_i}{V^i} \, | \, i \in \mathbb{N}\}$. 
\end{remark}

\begin{remark}[Non-symmetric multi-relations and hypergraphs]
A non-symmetric multi relation can be modeled by a hypergraph. The constituent function of a hypergraph chooses a non symmetric multi-relation on $V$, i.e. a subset of $\mathbb{O}(V) = \coprod_{n \geq 1} V^{n}$.
\end{remark}

\begin{definition}[\textbf{Order of an edge in a hypergraph}]
The \emph{order} of an edge $e$, written $|e|$ is defined as the cardinality of $f(e)$.
\end{definition}

\begin{definition}[\textbf{Undirected Hypergraph}]
An \emph{undirected hypergraph} is a triple $X=(X_E, X_V, f)$ where \onearrow{X_E}{f}{\mathbb{P}_{+} (X_V)}.
\end{definition}

\begin{definition}[$k$-\textbf{Uniform Hypergraph}]
A $k$-\emph{uniform hypergraph} is a hypergraph $X=(X_E, X_V, f)$ where $\forall e \in X_E \, . \, |e| = k$.
\end{definition}

\begin{definition}[\textbf{Directed Hypergraph Morphism}]
Let $X=\onearrow{X_E}{f}{\mathbb{O}_{+} (X_V)}$ and \onearrow{Y_E}{g}{\mathbb{O}_{+} (Y_V)}
be directed hypergraphs. A \emph{directed hypergraph morphism} is a pair of
functions \onearrow{X_E}{a}{Y_E} and \onearrow{X_V}{b}{Y_V} such that the following diagram commutes:
$$
\xymatrix {
X_E      \ar[r]^-{f} \ar[d]_{a}   &     {\mathbb{O}_{+} (X_V) = \coprod_{n \geq 1} X_V^n} \ar[d]^{b} \\
Y_E      \ar[r]^-{g}     &  \mathbb{O}_{+} (Y_V) = \coprod_{n \geq 1} Y_V^n
}
$$
\end{definition}

\begin{definition}[\textbf{Undirected Hypergraph Morphism}]
Let $X=\onearrow{X_E}{f}{\mathbb{P}_{+} (X_V)}$ and \onearrow{Y_E}{g}{\mathbb{P}_{+} (Y_V)} be directed hypergraphs. A \emph{undirected hypergraph morphism} is a pair of functions \onearrow{X_E}{a}{Y_E} and \onearrow{X_V}{b}{Y_V} such that the following diagram commutes:
$$
\xymatrix{
X_E  \ar[d]^a \ar[r]^f  & \mathbb{P}_{+} (X_V) \ar[d]^b \\ 
Y_E          \ar[r]^g       & \mathbb{P}_{+} (Y_V)
}
$$
\end{definition}

\begin{definition}[\textbf{Index category $\Delta_H$ for directed hypergraphs}]
The index category $\Delta_H$ for hypergraphs is defined by
\begin{description}
\item [objects] $\{V, E_1, E_2, E_3, \ldots \}$
\item [arrows] $\bigcup_{n \in \mathbb{N}} \{ \onearrow{V}{v_i^n}{E_n} \, | \, 0 \leq i \leq n \}$.
\end{description}
\end{definition}

\begin{proposition}[\textbf{Category \calH of directed hypergraphs}]
$\calH = \mathbf{Sets}^{\Delta_H^{\text{op}}} = \Delta_H^{\text{op}} \rightarrow \mathbf{Sets}$. The vertex set is $F_V = F(V)$. The hyperedge set is $F_E = \coprod_{n \in \mathbb{N}} {F(E_n)}$. The constituent functions are $\{ \onearrow{F(E_n)}{F(V_i)}{F(V)}\}_{n \in \mathbb{N}}$. Let \onearrow{\Delta_H^{\text{op}}}{F}{\mathbf{Sets}} be an element of \calH and \onearrow{\Delta_H^{\text{op}}}{F'}{\mathbf{Sets}} be another element of \calH.
$$
\xymatrix{
F  \ar[d]^{\eta} & {\mathbb{O}_{+} (E) = \coprod_{n}E_n} \ar[r]^-{v_i^n} \ar[d]^{\eta_E} & V \ar[d]^{\eta_V} \\
F'                      & {\mathbb{O}_{+} (E') = \coprod_{n}E'_n} \ar[r]^-{{v'}_i^n}                         & V' \\
}
$$
\end{proposition}

%%----------------------------------------------------------------------------------------------------------
\subsection{Semi-Simplicial Sets}

\begin{definition}[\textbf{Semi-Simplicial Set [Set theoretic version)}]
A \emph{semi-simplicial set} is a sequence $[X_n]_{n \in \mathbb{N}}$ of sets together with a collection $\{ \onearrow{X_n}{d_i^n}{X_{n-1}} \, | \, 0 \leq i \leq n \}$ of functions for each $n$ satisfying the identity
$$
d_i^n \circ d_j^{n+1} = d_{j-1}^n \circ d_i^{n+1}
$$
\end{definition}

\begin{remark}[Diagram for Simplicial Sets]
The set $X_n$ is called the set of $n$-\emph{simplices} of $X$. The function \onearrow{X_n}{d_i^n}{X_{n-1}} is often abbreviated $d_i$. The set of functions $d_i$ determines the lower dimensional simplices that form the boundary of the higher dimensional simplex.
$$ \xymatrix{
\ldots \ar@<1.25ex>[r]  \ar@<0.5ex>[r]  \ar@<-0.25ex>[r]  \ar@<-1.0ex>[r] &
     X_2 \ar@<0.75ex>[r]  \ar[r] \ar@<-0.75ex>[r] &
     X_1 \ar@<0.4ex>[r] \ar@<-0.4ex>[r] &
     X_0
} $$
\end{remark}

\begin{definition}[\textbf{1-skeleton}]
Let $X = 
\xymatrix{
\ldots \ar@<1.25ex>[r]  \ar@<0.5ex>[r]  \ar@<-0.25ex>[r]  \ar@<-1.0ex>[r] &
     X_2 \ar@<0.75ex>[r]  \ar[r] \ar@<-0.75ex>[r] &
     X_1 \ar@<0.4ex>[r]^{d_1} \ar@<-0.4ex>[r]_{d_0} &
     X_0
}
$ be a semi-simplicial set. Then the \emph{1-skeleton} of $X$ is the directed graph
$\xymatrix{
     X_1 \ar@<0.6ex>[r]^-{d_1} \ar@<-0.6ex>[r]_-{d_0} & X_0
}$
or equivalently
$\xymatrix{
     X_1 \ar[r]^-{d_1 \times d_0} & X_0 \times X_0
}$. This amounts to forgetting all the higher order simplicial sets and their associated functions.
\end{definition}

\begin{definition}[\textbf{morphism of semi-simplicial sets}]
A \emph{morphism of semi-simplicial sets} \onearrow{X}{\eta}{Y} is a collection of maps $\{ \onearrow{X_n}{\eta_n}{Y_n} \, | \,n \in \mathbb{N} \}$ such that $\forall i \, . \, 0 \leq i \leq n+1 \Longrightarrow \eta_n \circ d_i^{n+1} = e_i^{n+1} \circ \eta_{n+1}$
\end{definition}

$$
\xymatrix{
\ldots \ar@<3ex>[r]|{d_3^3}  \ar@<1 ex>[r]|{d_2^3}  \ar@<-1ex>[r]|{d_1^3}  \ar@<-3 ex>[r]|{d_0^3} &
     X_2 \ar@<2ex>[r]|{d_2^2}  \ar[r]|{d_1^2} \ar@<-2 ex>[r]|{d_0^2} \ar[dd]^{\eta_2} &
     X_1 \ar@<1ex>[r]|{d_1^1} \ar@<-1ex>[r]|{d_0^1} \ar[dd]^{\eta_1} &
     X_0 \ar[dd]^{\eta_0} \\
\ldots     \\
  \ldots \ar@<3ex>[r]|{e_3^3}  \ar@<1 ex>[r]|{e_2^3}  \ar@<-1ex>[r]|{d_1^3}  \ar@<-3 ex>[r]|{e_0^3} &
     Y_2 \ar@<2ex>[r]|{e_2^2}  \ar[r]|{e_1^2} \ar@<-2 ex>[r]|{e_0^2} &
     Y_1 \ar@<1ex>[r]|{e_1^1} \ar@<-1ex>[r]|{e_0^1} &
     Y_0
}
$$

\begin{definition}[\textbf{Index category $\Delta_{>}$ for semi-simplicial sets}]
The index category $\Delta_{>}$ is defined by
\begin{description}
\item [objects] The finite non-empty sets ${0,1, \ldots , n}$ for $n \geq 0$.
\item [arrows] Strictly increasing functions \onearrow{m}{f}{n}.
\end{description}
Let $\Delta_{>}^{\text{op}}$ denote the opposite category of $\Delta_{>}$.
\end{definition}

\begin{remark}
There are two strictly increasing maps from $\{0\}$ to $\{0, 1\}$, namely $0 \mapsto 0$ and $0 \mapsto 1$. There are three strictly increasing maps from $\{0, 1\}$ to $\{0, 1, 2\}$, they are $\{ 0 \mapsto 0, 1 \mapsto 1\}$, $\{ 0 \mapsto 0, 1 \mapsto 2\}$ and $\{ 0 \mapsto 1, 1 \mapsto 2\}$. In general there are $n+1$ possible maps from $\{0, \ldots n-1\}$ to $\{0, \ldots n \}$, based on skipping one of the elements in $\{0, \ldots n \}$. We can call these maps $\lambda^n_i$, where $i$ is the number skipped. Therefore we can diagram $\Delta_{>}^{\text{op}}$ as
$$
\xymatrix{
\ldots \ar@<2.1ex>[r]  \ar@<0.7 ex>[r]  \ar@<-0.7ex>[r]  \ar@<-2.1 ex>[r] &
     2 \ar@<1.2ex>[r]  \ar[r] \ar@<-1.2 ex>[r] &
     1 \ar@<0.6ex>[r] \ar@<-0.6ex>[r] &
     0
}
$$
\end{remark}

\begin{proposition}[\textbf{Category \calS of semi simplicial sets}]
$\calS = \mathbf{Sets}^{\Delta_{>}^{\text{op}}} = \Delta_{>}^{\text{op}} \rightarrow \mathbf{Sets}$. Let $F \in \calS$,
so that \onearrow{ \Delta_{>}^{\text{op}}}{F}{\mathbf{Sets}}, and also let Let $F' \in \calS$,
so that \onearrow{ \Delta_{>}^{\text{op}}}{F'}{\mathbf{Sets}}.
Based on the shape of the category $\Delta_{>}^{\text{op}}$ shown in the previous remark we see that a natural transformation \onearrow{F}{\eta}{F'} produces the information for a morphism between semi-simplicial sets.
For example, $F(\{0,1,2\}) = X_2$ and $F'(\{0,1,2\}) = Y_2$ and $F(\lambda^2_1) = d^2_1$ and $F'(\lambda^2_1) = e^2_1$, therefore the naturality of $\eta$ imposes that $\eta_1 \circ  d^2_1 = e^2_1 \circ \eta_2$.
And in general, naturality of $\eta$ imposes $\eta_n \circ  d^n_i = e^n_i \circ \eta_{n+1}$
which returns the definition of a morphism of semi-simplicial sets.
\end{proposition}

%%----------------------------------------------------------------------------------------------------------
\subsection{Simplicial Sets}

\begin{definition}[\textbf{Index category $\Delta_{>}$ for Simplicial sets}]
The index category $\Delta_{\geq}$ is defined by
\begin{description}
\item [objects] The finite non-empty sets ${0,1, \ldots , n}$ for $n \geq 0$. The cardinality of the set is $n$, but the set need not contain the numbers from $0$ through $n$. Thus the objects are strictly increasing sequences of ordinal numbers.
These objects have internal order, they can also be written as $0 \rightarrow 1 \rightarrow \ldots \rightarrow n$.
\item [arrows] Non-decreasing (also called \emph{increasing}) functions \onearrow{m}{f}{n}.
\end{description}
Let $\Delta_{\geq}^{\text{op}}$ denote the opposite category of $\Delta_{\geq}$.
The category $\Delta_{\geq}$ is also called the \emph{simplicial category}.
\end{definition}

\begin{definition}[\textbf{Category \calS of simplicial sets}]
The category of simplicial sets is $\mathbf{sSet} = \mathbf{Sets}^{\Delta_{\geq}^{\text{op}}} = \Delta_{\geq}^{\text{op}} \rightarrow \mathbf{Sets}$. Equivalently, we can use a contra-variant functor from $\Delta_{\geq}^{\text{op}}$ to $\mathbf{Sets}$.
\end{definition}

\begin{remark}[Semi-simplicial and simplicial sets]
The big difference between \emph{simplicial} and \emph{semi-simplicial} sets is the existence of non-injective morphisms in $\Delta_{\geq}$. This permits representing simplices (triangles) that are \emph{degenerate}. A degenerate $n$-simplex has been flattened to one of its $n-k$ dimensional faces.
\end{remark}

\begin{definition}[\textbf{Face maps in} $\Delta_{\geq}$]
The \emph{face map} $\onearrow{n}{d_i^n}{n-1}$ in $\Delta_{\geq}^{\text{op}}$ is defined by
$$
d_i^n (0 \rightarrow \ldots \rightarrow n) = (0 \rightarrow \ldots i - 1 \rightarrow i + 1 \rightarrow \ldots n)
$$
\end{definition}

\begin{definition}[\textbf{Degeneracy maps in} $\Delta_{\geq}$]
The \emph{degeneracy map} $\onearrow{n}{s_i^n}{n-1}$ in $\Delta_{\geq}^{\text{op}}$ is defined by
$$
s_i^n (0 \rightarrow \ldots \rightarrow n) = (0 \rightarrow \ldots i - 1 \rightarrow i - 1 \rightarrow i + 1 \rightarrow \ldots n)
$$
\end{definition}

\begin{proposition}[\textbf{Simplicial Identities}]
The face maps and degeneracy maps satisfy the following definitions.
$$
\xymatrix@R=1mm@C=0.6mm{
 & (1) & d_i d_j & = & d_{j-1} d_i & \text{if} & i < j \\
 & (2) & d_i s_j & = & s_{j-1} d_i  & \text{if} & i < j \\
 & (3) & d_j s_j & = & d_{j+1} s_j & = \text{id}  \\
 & (4) & d_i s_j & = & s_j d_{i-1} & \text{if} & i > j + 1 \\
 & (5) & s_i s_j & = & s_{j+1} s_i & \text{if} & i \leq j
}
$$
\end{proposition}

\begin{definition}[\textbf{Standard} $n$-\textbf{simplex}]
The \emph{standard} $n$-\emph{simplex}, written $\Delta^{n}$ is the functor \onearrow{\Delta}{\hom(-,n)}{\Delta} defined as follows:
\begin{description}
\item [objects] \maptob{\hom(-,n)}{m}{\hom(m,n)}
\item[arrows]  Let \onearrow{j}{\alpha}{k} be a morphism. Then \maptob{\hom(-,n)}{\onearrow{j}{\alpha}{k}}{\hom(\alpha,n)}, where \onearrow{\hom(k,n)}{\hom(\alpha,n)}{\hom(j,n)} is given by \maptob{\hom(\alpha,n)}{g}{g \circ h}.
\end{description}

\end{definition}

%%----------------------------------------------------------------------------------------------------------
\subsection{Symmetric Simplicial Sets}

\begin{definition}[Symmetric Simplicial Sets]
Let $\Sigma$ be the category defined as
\begin{description}
\item [objects] Sets $\mathbf{n} = \{0, \ldots, n \}$
\item [morphisms] The functions \onearrow{\mathbf{m}}{}{\mathbf{n}}. There are $(n + 1)^{m+1}$ of these morphisms.
\end{description}
\end{definition}

\begin{remark}[Visualization of Symmetric Simplicial Sets]
Visualize as a simplicial set in which the vertices of an edge are unordered.
\end{remark}

%%----------------------------------------------------------------------------------------------------------
\subsection{\textbf{Graphs vs Hypergraphs}}

The category of graphs is $\calG = \mathbf{Sets}^{\Delta_G^{\text{op}}} =  \Delta_G^{\text{op}} \rightarrow \mathbf{Sets}$, while the category of hypergraphs is
$\calH = \mathbf{Sets}^{\Delta_H^{\text{op}}} = \Delta_H^{\text{op}} \rightarrow \mathbf{Sets}$.
$$
\xymatrix{
V \ar@{|->}[d]_{\top} & E_2 \ar@{|->}[d]_{\top} &\Delta_G \ar[d]_{\top} & \calG \ar@/^1pc/[d]|{L_i} \\
V                                 & E_2                                  & \Delta_H                       & \calH \ar@/^1pc/[u]|{R_i}
}
$$
$R_i$ takes a graph to itself. $L_i$ forgets everything except vertices and 2-edges.

%%----------------------------------------------------------------------------------------------------------
\subsection{Hypergraph vs. Simplicial Sets}
Graphs, hypergraphs, semi-simplicial sets and simplicial sets are all categories of functors from some index category to $\mathbf{Sets}$. Construct a functor \onearrow{\Delta_H}{a}{\Delta_{\leq}} as follows:
\begin{description}
\item [objects] \mapto{V}{[0]} and \mapto{E_n}{[n-1]}
\item [arrows] For each $1 \leq i \leq n$ map the morphism \onearrow{V}{v_i^n}{E_n} to the morphism (denoted \mapto{0}{i-1})which is defined by \maptob{\mapto{0}{i-1}}{[0]}{[n-1]}.
\end{description}
Construct the opposite functor \onearrow{\Delta_H^{\text{op}}}{A}{\Delta_{\leq}^{\text{op}}}. We get the functors
$$\xymatrix{
\calH = \mathbf{Sets}^{\Delta_H^{\text{op}}} \ar@<0.8ex>[r]|{L_A} & \mathbf{Sets}^{\Delta_{>}^{\text{op}}} = \calS \ar@<0.8ex>[l]|{R_A}
}$$
The functor $R_A$ forgets all relationships between $n$-simplices and $m$-simplices unless $m = 0$. The functor $L_A$ takes a hypergraph $H$ and sends it to a simplicial set with 0-simplices that are the vertices of $H$ and with $n$-simplices that are the $n+1$ edges of $H$.

\begin{definition}[$R_A$ and $L_A$]
$$
\xymatrix{
\calC \ar[d]_{A}    & G \circ A \ar@{(->}[r] \ar@{|->}@<1ex>[d]^{L_A} & \mathbf{Sets}^{\calD} \ar@<1ex>[d]^{L_A} \\
\calD                     & G             \ar@{(->}[r] \ar@{|->}@<1ex>[u]^{R_A} & \mathbf{Sets}^{\calC} \ar@<1ex>[u]^{R_A}
}
$$
There is a natural bijection
\xymatrix{
\text{Home}_{\mathbf{Sets}^{\calC}} (F, R_A (G)) \ar[r]^{\equiv} & \text{Hom}_{\mathbf{Sets}^{\calD}} (L_A (F), G) 
}
\end{definition}

\section{Category of Higher Graphs}

\begin{definition}[\textbf{Basic indexing category for higher graphs}]
Define the category $\Delta_B$ as follows
\begin{description}
\item [objects] $V$ and $E_n$ for $ n \geq 2$.
\item [morphisms] \onearrow{V}{V_i^n}{E_n} for every $1 \leq i \leq n$.
\end{description}
\end{definition}

\begin{definition}[\textbf{Indexing category for higher graphs}]
A pair $(\calI, \onearrow{\Delta_B}{F}{\calI})$ where \calI is a category and $F$ is a functor. When $F$ is faithful, the category is called a \emph{faithful} indexing category for higher graphs.
\end{definition}

\begin{remark}
An indexing category is a category that has a reasonable interpretation as vertices and $n$-edges. When $F$ is a faithful functor the $n$ vertices of an $n$-edge are distinct. $\Delta_H$, $\Delta_{\>}$, and
$\Delta_{\geq}$ are faithful indexing categories for higher graphs. $\Delta_G$ is an indexing category for higher graphs, but not a faithful indexing category for higher graphs, thus graphs are not a faithful representation of higher graphs.
\end{remark}

\begin{definition}[\textbf{Basic Category of Higher Graphs}]
Let $\calB = \Delta_B \rightarrow \mathbf{Sets} = \mathbf{Sets}^{\Delta_B^{\text{op}}}$. \calB is the \emph{basic category of higher graphs}. The objects of \calB are \emph{basic higher graphs}.
\end{definition}

\begin{definition}[\textbf{Category of Higher Graphs}]
A category of the form $\mathbf{Sets}^{{\calI}^{\text{op}}}$ where $(\calI, \onearrow{\calB}{f}{\calI})$ is an indexing category for higher graphs. The category is \emph{faithful} when $f$ is faithful. $\mathbf{Sets}^{{\calI}^{\text{op}}}$ is called the \emph{category of \calI-indexed graphs}.
\end{definition}

\begin{remark}
Given an indexing category for higher graphs $(\calI, \onearrow{\Delta_B}{f}{\calI})$ we get the pair of functors
$\xymatrix{
\calB = \mathbf{Sets}^{\Delta_B^{\text{op}}} \ar@<0.8ex>[r]|(0.6){L_f} & \mathbf{Sets}^{\calI^{\text{op}}} \ar@<0.8ex>[l]|(0.4){R_f}
}$ $R_f$ produces the \emph{basic underlying graph}.
\end{remark}

%%----------------------------------------------------------------------------------------------------------
%%----------------------------------------------------------------------------------------------------------

\section{Networks and their models}

%%----------------------------------------------------------------------------------------------------------
\subsection{Networks modeled by graphs}

Graphs work well as models for networks in which every connection is one-to-one and private. The following are examples:
\begin{description}
\item [Transportation networks] The vertices are locations, and the edges are roads.
\item [Tournaments] The vertices are players, the edges are matches.
\item [Electric circuits] The vertices are circuit elements, the edges are wires.
\item [Private gossip] The vertices are people, the edges are "private" conversations.
\end{description}

%%----------------------------------------------------------------------------------------------------------
\subsection{Networks modeled by hypergraphs}

When there are groups that interact in a way that subgroups of the group do not. An example is coauthorship of papers. If three people are coauthoring a paper, it does not imply that any two of them can be said to be "the coauthors of the paper".

\begin{remark}[Sub-edge problem for hypergraphs]
Let $X$ be the hypergraph with vertices $V = \{a, b, c, d\}$ and edges $e_1 = (a,b), e_2 = (a,b,c), e_3 = (a,b,c)$.
Then $e_1$ is either a subedge of $e_2$, or a subedge of $e_3$, or perhaps both. This ambiguity of is why hypergraphs are not good with common subedges. Looking at the index category $\Delta_H$ for hypergraphs,
there are no morphisms \onearrow{E_n}{}{E_m} when $m \neq n$. Therefore a hypergraph \onearrow{\Delta_H^{\text{op}}}{X}{\mathbf{Sets}}, so a hypergraph cannot compare the set $X(E_m)$ of $m$-edges with the set $X(E_n)$ of $n$-edges.
\end{remark}

\begin{example}[Another look at the sub-edge problem for hypergraphs]
Let $X'$ be a hypergraph with the same vertices as $X$ above, but with edges $e'_1 = (a,b), e'_2 = (b,c), e'_3 = (a,b,c)$.
We \emph{cannot} construct a morphism from $X'$ to $X$. For example, if \mapto{e'_1}{e_1}, then we need \mapto{a}{a} and \mapto{b}{b}. Next we need to map \mapto{e'_2}{e_1}, because $e_1$ is the only vertex in $X$ with two vertices. This requires \mapto{b}{a} and \mapto{c}{b}, so both sets of constraints cannot be satisfied at the same time.
Thus the event of user $a$ joining group $e'_2$ cannot be modeled.
\end{example}

%%----------------------------------------------------------------------------------------------------------
\subsection{Networks modeled by semi-simplicial sets}
When a multi-relation is closed under taking subsets, a semi-simplicial set is often a better model of the network than a hypergraph. Hypergraphs do not have a useful notion of subedges, but semi-simplicial sets have a good notion of subedges.

\begin{example}
Consider a set of interacting processes. Each process has a finite number of parts which must occur in a fixed order.
Sometimes a process $P$ can be a subprocess of two other processes $Q$ and $R$. In modeling this as a semi-simplicial set the processes $P$ and $Q$ are glued along $P$.
\end{example}

\subsection{Category of preorders as an indexing category}

\end{document}
