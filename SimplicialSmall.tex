\documentclass[10pt]{article}


\newcommand{\onearrow}[3]{\mbox{$#1 \stackrel{#2}{\longrightarrow} #3$}}
\newcommand{\calB}{\mbox{${\cal B}\ $}}
\newcommand{\calC}{\mbox{${\cal C}\ $}}
\newcommand{\calD}{\mbox{${\cal D}\ $}}
\newcommand{\calG}{\mbox{${\cal G}\ $}}
\newcommand{\calH}{\mbox{${\cal H}\ $}}
\newcommand{\calI}{\mbox{${\cal I}\ $}}
\newcommand{\calS}{\mbox{${\cal S}\ $}}
\newcommand{\functor}[3]{\mbox{${\cal #1} \stackrel{#2}{\rightarrow}{\cal #3}$}}
\newcommand{\mapto}[2]{\mbox{$#1 \mapsto #2$}}
\newcommand{\maptob}[3]{\mbox{$#1: #2 \mapsto #3$}}

\usepackage{amsthm}
\usepackage{amssymb}
\usepackage{amsmath,amscd}
\usepackage[all,cmtip]{xy}
\theoremstyle{remark}
\newtheorem{remark}{Remark}
\newtheorem{definition}{Definition}
\newtheorem{proposition}{Proposition}
\newtheorem{example}{Example}

\begin{document}

\section{Graphs, Hypergraphs and Simplicial Sets}

%%----------------------------------------------------------------------------------------------------------
\subsection{Examples}

$$
\xymatrix{
\{abcd\} \ar@<3ex>[r]|{d_3^3}  \ar@<1 ex>[r]|{d_2^3}  \ar@<-1ex>[r]|{d_1^3}  \ar@<-3 ex>[r]|{d_0^3} &
     \left\{ {\begin{array}{c} abc \\ abd \\ acd \\ bcd \end{array}} \right\} \ar@<2ex>[r]|{d_2^2}  \ar[r]|{d_1^2} \ar@<-2 ex>[r]|{d_0^2} &
     \left\{ {\begin{array}{c} ab \\ ac \\ ad \\ bc \\ bd \\ cd \end{array}} \right\} \ar@<1ex>[r]|{d_1^1} \ar@<-1ex>[r]|{d_0^1} &
     \left\{ {\begin{array}{c} a \\ b \\ c \\ d \end{array}} \right\}
}
$$

$$
\xymatrix{
     \left\{ {\begin{array}{c} abc \\ bcd \end{array}} \right\} \ar@<2ex>[r]|{d_2^2}  \ar[r]|{d_1^2} \ar@<-2 ex>[r]|{d_0^2} &
     \left\{ {\begin{array}{c} ab \\ ac \\ bc \\ bd \\ cd \end{array}} \right\} \ar@<1ex>[r]|{d_1^1} \ar@<-1ex>[r]|{d_0^1} &
     \left\{ {\begin{array}{c} a \\ b \\ c \\ d \end{array}} \right\}
}
$$

$$
\xymatrix{
\{ABCD\} \ar@<3ex>[r]|{d_3^3}  \ar@<1 ex>[r]|{d_2^3}  \ar@<-1ex>[r]|{d_1^3}  \ar@<-3 ex>[r]|{d_0^3} &
     \left\{ {\begin{array}{c} ABC \\ ABD \\ ACD \\ BCD \end{array}} \right\} \ar@<2ex>[r]|{d_2^2}  \ar[r]|{d_1^2} \ar@<-2 ex>[r]|{d_0^2} &
     \left\{ {\begin{array}{c} AB \\ AC \\ AD \\ BC \\ BD \\ CD \end{array}} \right\} \ar@<1ex>[r]|{d_1^1} \ar@<-1ex>[r]|{d_0^1} &
     \left\{ {\begin{array}{c} A \\ B \\ C \\ D \end{array}} \right\}
}
$$

$$
\xymatrix{
     \left\{ {\begin{array}{c} ABC \\ BCD \end{array}} \right\} \ar@<2ex>[r]|{d_2^2}  \ar[r]|{d_1^2} \ar@<-2 ex>[r]|{d_0^2} &
     \left\{ {\begin{array}{c} AB \\ AC \\ BC \\ BD \\ CD \end{array}} \right\} \ar@<1ex>[r]|{d_1^1} \ar@<-1ex>[r]|{d_0^1} &
     \left\{ {\begin{array}{c} A \\ B \\ C \\ D \end{array}} \right\}
}
$$


%%----------------------------------------------------------------------------------------------------------

\subsection{Semi-Simplicial Sets}

\begin{definition}[\textbf{Semi-Simplicial Set [Set theoretic version)}]
A \emph{semi-simplicial set} is a sequence $[X_n]_{n \in \mathbb{N}}$ of sets together with a collection $\{ \onearrow{X_n}{d_i^n}{X_{n-1}} \, | \, 0 \leq i \leq n \}$ of functions for each $n$ satisfying the identity
$$
d_i^n \circ d_j^{n+1} = d_{j-1}^n \circ d_i^{n+1}
$$
\end{definition}

\begin{remark}[Diagram for Simplicial Sets]
The set $X_n$ is called the set of $n$-\emph{simplices} of $X$. The function \onearrow{X_n}{d_i^n}{X_{n-1}} is often abbreviated $d_i$. The set of functions $d_i$ determines the lower dimensional simplices that form the boundary of the higher dimensional simplex.
$$ \xymatrix{
\ldots \ar@<1.25ex>[r]  \ar@<0.5ex>[r]  \ar@<-0.25ex>[r]  \ar@<-1.0ex>[r] &
     X_2 \ar@<0.75ex>[r]  \ar[r] \ar@<-0.75ex>[r] &
     X_1 \ar@<0.4ex>[r] \ar@<-0.4ex>[r] &
     X_0
} $$
\end{remark}

\begin{definition}[\textbf{1-skeleton}]
Let $X = 
\xymatrix{
\ldots \ar@<1.25ex>[r]  \ar@<0.5ex>[r]  \ar@<-0.25ex>[r]  \ar@<-1.0ex>[r] &
     X_2 \ar@<0.75ex>[r]  \ar[r] \ar@<-0.75ex>[r] &
     X_1 \ar@<0.4ex>[r]^{d_1} \ar@<-0.4ex>[r]_{d_0} &
     X_0
}
$ be a semi-simplicial set. Then the \emph{1-skeleton} of $X$ is the directed graph
$\xymatrix{
     X_1 \ar@<0.6ex>[r]^-{d_1} \ar@<-0.6ex>[r]_-{d_0} & X_0
}$
or equivalently
$\xymatrix{
     X_1 \ar[r]^-{d_1 \times d_0} & X_0 \times X_0
}$. This amounts to forgetting all the higher order simplicial sets and their associated functions.
\end{definition}

\begin{definition}[\textbf{morphism of semi-simplicial sets}]
A \emph{morphism of semi-simplicial sets} \onearrow{X}{\eta}{Y} is a collection of maps $\{ \onearrow{X_n}{\eta_n}{Y_n} \, | \,n \in \mathbb{N} \}$ such that $\forall i \, . \, 0 \leq i \leq n+1 \Longrightarrow \eta_n \circ d_i^{n+1} = e_i^{n+1} \circ \eta_{n+1}$
\end{definition}

$$
\xymatrix{
\ldots \ar@<3ex>[r]|{d_3^3}  \ar@<1 ex>[r]|{d_2^3}  \ar@<-1ex>[r]|{d_1^3}  \ar@<-3 ex>[r]|{d_0^3} &
     X_2 \ar@<2ex>[r]|{d_2^2}  \ar[r]|{d_1^2} \ar@<-2 ex>[r]|{d_0^2} \ar[dd]^{\eta_2} &
     X_1 \ar@<1ex>[r]|{d_1^1} \ar@<-1ex>[r]|{d_0^1} \ar[dd]^{\eta_1} &
     X_0 \ar[dd]^{\eta_0} \\
\ldots     \\
  \ldots \ar@<3ex>[r]|{e_3^3}  \ar@<1 ex>[r]|{e_2^3}  \ar@<-1ex>[r]|{d_1^3}  \ar@<-3 ex>[r]|{e_0^3} &
     Y_2 \ar@<2ex>[r]|{e_2^2}  \ar[r]|{e_1^2} \ar@<-2 ex>[r]|{e_0^2} &
     Y_1 \ar@<1ex>[r]|{e_1^1} \ar@<-1ex>[r]|{e_0^1} &
     Y_0
}
$$

\begin{definition}[\textbf{Index category $\Delta_{>}$ for semi-simplicial sets}]
The index category $\Delta_{>}$ is defined by
\begin{description}
\item [objects] The finite non-empty sets ${0,1, \ldots , n}$ for $n \geq 0$.
\item [arrows] Strictly increasing functions \onearrow{m}{f}{n}.
\end{description}
Let $\Delta_{>}^{\text{op}}$ denote the opposite category of $\Delta_{>}$.
\end{definition}

\begin{remark}
There are two strictly increasing maps from $\{0\}$ to $\{0, 1\}$, namely $0 \mapsto 0$ and $0 \mapsto 1$. There are three strictly increasing maps from $\{0, 1\}$ to $\{0, 1, 2\}$, they are $\{ 0 \mapsto 0, 1 \mapsto 1\}$, $\{ 0 \mapsto 0, 1 \mapsto 2\}$ and $\{ 0 \mapsto 1, 1 \mapsto 2\}$. In general there are $n+1$ possible maps from $\{0, \ldots n-1\}$ to $\{0, \ldots n \}$, based on skipping one of the elements in $\{0, \ldots n \}$. We can call these maps $\lambda^n_i$, where $i$ is the number skipped. Therefore we can diagram $\Delta_{>}^{\text{op}}$ as
$$
\xymatrix{
\ldots \ar@<2.1ex>[r]  \ar@<0.7 ex>[r]  \ar@<-0.7ex>[r]  \ar@<-2.1 ex>[r] &
     2 \ar@<1.2ex>[r]  \ar[r] \ar@<-1.2 ex>[r] &
     1 \ar@<0.6ex>[r] \ar@<-0.6ex>[r] &
     0
}
$$
\end{remark}

\begin{proposition}[\textbf{Category \calS of semi simplicial sets}]
$\calS = \mathbf{Sets}^{\Delta_{>}^{\text{op}}} = \Delta_{>}^{\text{op}} \rightarrow \mathbf{Sets}$. Let $F \in \calS$,
so that \onearrow{ \Delta_{>}^{\text{op}}}{F}{\mathbf{Sets}}, and also let Let $F' \in \calS$,
so that \onearrow{ \Delta_{>}^{\text{op}}}{F'}{\mathbf{Sets}}.
Based on the shape of the category $\Delta_{>}^{\text{op}}$ shown in the previous remark we see that a natural transformation \onearrow{F}{\eta}{F'} produces the information for a morphism between semi-simplicial sets.
For example, $F(\{0,1,2\}) = X_2$ and $F'(\{0,1,2\}) = Y_2$ and $F(\lambda^2_1) = d^2_1$ and $F'(\lambda^2_1) = e^2_1$, therefore the naturality of $\eta$ imposes that $\eta_1 \circ  d^2_1 = e^2_1 \circ \eta_2$.
And in general, naturality of $\eta$ imposes $\eta_n \circ  d^n_i = e^n_i \circ \eta_{n+1}$
which returns the definition of a morphism of semi-simplicial sets.
\end{proposition}

%%----------------------------------------------------------------------------------------------------------
\subsection{Simplicial Sets}

\begin{definition}[\textbf{Index category $\Delta_{>}$ for Simplicial sets}]
The index category $\Delta_{\geq}$ is defined by
\begin{description}
\item [objects] The finite non-empty sets ${0,1, \ldots , n}$ for $n \geq 0$. The cardinality of the set is $n$, but the set need not contain the numbers from $0$ through $n$. Thus the objects are strictly increasing sequences of ordinal numbers.
These objects have internal order, they can also be written as $0 \rightarrow 1 \rightarrow \ldots \rightarrow n$.
\item [arrows] Non-decreasing (also called \emph{increasing}) functions \onearrow{m}{f}{n}.
\end{description}
Let $\Delta_{\geq}^{\text{op}}$ denote the opposite category of $\Delta_{\geq}$.
The category $\Delta_{\geq}$ is also called the \emph{simplicial category}.
\end{definition}

\begin{definition}[\textbf{Category \calS of simplicial sets}]
The category of simplicial sets is $\mathbf{sSet} = \mathbf{Sets}^{\Delta_{\geq}^{\text{op}}} = \Delta_{\geq}^{\text{op}} \rightarrow \mathbf{Sets}$. Equivalently, we can use a contra-variant functor from $\Delta_{\geq}^{\text{op}}$ to $\mathbf{Sets}$.
\end{definition}

\begin{remark}[Semi-simplicial and simplicial sets]
The big difference between \emph{simplicial} and \emph{semi-simplicial} sets is the existence of non-injective morphisms in $\Delta_{\geq}$. This permits representing simplices (triangles) that are \emph{degenerate}. A degenerate $n$-simplex has been flattened to one of its $n-k$ dimensional faces.
\end{remark}

\begin{definition}[\textbf{Face maps in} $\Delta_{\geq}$]
The \emph{face map} $\onearrow{n}{d_i^n}{n-1}$ in $\Delta_{\geq}^{\text{op}}$ is defined by
$$
d_i^n (0 \rightarrow \ldots \rightarrow n) = (0 \rightarrow \ldots i - 1 \rightarrow i + 1 \rightarrow \ldots n)
$$
\end{definition}

\begin{definition}[\textbf{Degeneracy maps in} $\Delta_{\geq}$]
The \emph{degeneracy map} $\onearrow{n}{s_i^n}{n-1}$ in $\Delta_{\geq}^{\text{op}}$ is defined by
$$
s_i^n (0 \rightarrow \ldots \rightarrow n) = (0 \rightarrow \ldots i - 1 \rightarrow i - 1 \rightarrow i + 1 \rightarrow \ldots n)
$$
\end{definition}

\begin{proposition}[\textbf{Simplicial Identities}]
The face maps and degeneracy maps satisfy the following definitions.
$$
\xymatrix@R=1mm@C=0.6mm{
 & (1) & d_i d_j & = & d_{j-1} d_i & \text{if} & i < j \\
 & (2) & d_i s_j & = & s_{j-1} d_i  & \text{if} & i < j \\
 & (3) & d_j s_j & = & d_{j+1} s_j & = \text{id}  \\
 & (4) & d_i s_j & = & s_j d_{i-1} & \text{if} & i > j + 1 \\
 & (5) & s_i s_j & = & s_{j+1} s_i & \text{if} & i \leq j
}
$$
\end{proposition}

\end{document}
